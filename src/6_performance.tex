\section{Performance of Our Driver}
\label{chap:performance}

\subsection{First Experiments}
\label{chap:performance.firstexperiments}

\subsection{Final Experimental Setup}
\label{chap:performance.finalsetup}

\subsection{Results}
\label{chap:performance.results}
For testing purposes, we tried optimizing the performance by restricting support to AES256-XTS. This enabled removing some if-else constructs that dispatched de-/encryption functions based on whether AES128 or AES256 was used. Even though these conditionals were located in a performance critical section, we saw no speed improvements. Our theory for why this was the case is the following: our driver was only ever used for one LUKS2 partition and therefore always took the same path through the if-else (either always AES128 or always AES256). This trained the CPU's branch prediction on this one specific path. Thus, after a short training phase, the CPU always speculatively executed the correct path, resulting in the same performance as without the if-else.

The default compiler optimization settings in Visual Studio were a little bit conservative and also optimized for smaller code size than more speed. After tuning these settings to enable more aggressive optimizations and also focus on speed rather than code size, we found that \todo{this is a little bit complicated...}.