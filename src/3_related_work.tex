\section{Related Work}
\label{chap:relatedwork}
\todo{todo}

\subsection{Measuring Filesystem Driver Performance}
\label{chap:relatedwork.fsperformance}
In this section, we will layout possibilities for measuring performance on the file system level \todo{as well as present criteria for good benchmarks}. This is the theoretical foundation for most of the experiments described in \autoref{chap:performance}.

\cite{Traeger2008}: criteria for good fs benchmarks, mentions encrypted storage explicitly

\cite{Wright2003}: performance study of file encryption systems

\cite{Agrawal2008}: SSD workload and performance examination

\cite{Yadgar2021}: also SSD-specific, Table 1 has a good overview of ``trace repository'' papers, conclusion has a paragraph on the validity of using old workload traces

\cite{He2017}: also SSD-specific, description of different workloads (not sure how useful, though), developed a new simulator

\cite{Riska2006}, \cite{Riska2006a}, \cite{Leung2008}, \cite{Riska2009}, and \cite{Sehgal2010}: characterization/description of different disk drive workloads (maybe look at what Phoronix Suite/Postmark uses?)

\cite{Seo2014}: explains IO traces, defines classes of traces, and classifies real world traces

\cite{Pereira2013}: critically discusses IO traces and their replays

\cite{Ruemmler1993}: describes three (old) UNIX traces

\subsection{Cryptographic Aspects of LUKS2}
\label{chap:relatedwork.cryptography}
\todo{\cite{Fruhwirth2005}}

\todo{Search for more papers, e.g. attacks against LUKS? e.g. \cite{Polasek2019}}